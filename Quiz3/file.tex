\documentclass[12pt]{article}
\author{Lawrence Liu}
\usepackage{subcaption}
\usepackage{graphicx}
\usepackage{amsmath}
\usepackage{pdfpages}
\newcommand{\Laplace}{\mathscr{L}}
\setlength{\parskip}{\baselineskip}%
\setlength{\parindent}{0pt}%
\usepackage{xcolor}
\usepackage{listings}
\definecolor{backcolour}{rgb}{0.95,0.95,0.92}
\usepackage{amssymb}
\usepackage{empheq}

\newcommand*\widefbox[1]{\fbox{\hspace{2em}#1\hspace{2em}}}
\lstdefinestyle{mystyle}{
    backgroundcolor=\color{backcolour}}
\lstset{style=mystyle}

\title{ECE 131A Quiz 1}
\begin{document}
\maketitle
\section*{Problem 1}
For the 4 bears in the middle, the probability that the bear is happy is 
$\frac{2}{5}$ and for the bears on the side the probability that the bear is happy is
$\frac{1}{5}$, therefore the expected value of the number of happy bear pairs is
$$\frac{1}{2}(4\cdot\frac{2}{5}+\frac{1}{5}+\frac{1}{5})=\boxed{1}$$
\section*{Problem 2}
\subsection*{(a)}
This would be a hypergeometric distribution. The parameters would be
$K=7$, $N=1000$, $n=10$. So the pmf for the number of green envolopes
would be
$$p(k)=\frac{\binom{K}{k}\binom{N-K}{n-k}}{\binom{N}{n}}$$
\subsection*{(b)}
The probability of getting 1 envelope is:
$$p(1)=\frac{\binom{7}{1}\binom{993}{9}}{\binom{1000}{10}}=0.06629133689$$
\subsection*{(c)}
This would be a binomial distribution. The parameters would be
$N=10$ and $p=\frac{7}{1000}$, so the pmf for the number of green envolopes $k$ 
is:
$$p(k)=\binom{N}{k}p^k(1-p)^{N-k}$$
\subsection*{(d)}
The probability of getting 1 envelope is:
$$10\left(\frac{7}{1000}\right)\left(\frac{993}{1000}\right)^9=\boxed{0.0657}$$
\section*{Problem 3}
\subsection*{(a)}
The phone ringing, email, arriving, and next computer beeping are 
possion processes, so the probability of no phone ringing for the next 10 seconds
is 
$$\frac{(\frac{10}{30})^0}{0!}e^{-\frac{10}{30}} = e^{-\frac{1}{3}}$$
Likewise the probability of no email arriving for the next 10 seconds is
$$\frac{(\frac{10}{20})^0}{0!}e^{-\frac{10}{20}} = e^{-\frac{1}{2}}$$
And the probability of no computer beeping for the next 10 seconds is
$$\frac{(\frac{10}{15})^0}{0!}e^{-\frac{10}{15}} = e^{-\frac{2}{3}}$$
Thus the total probability of all of these events is 
$$e^{-\frac{1}{3}}e^{-\frac{1}{2}}e^{-\frac{2}{3}} = \boxed{e^{-\frac{3}{2}}}$$
\subsection*{(b)}
The probability of the phone ringing, email arriving, and computer beeping
are possion processes. So the probability of none of these events happening for 
$t$ seconds is:
$$e^{-\frac{t}{30}}e^{-\frac{t}{20}}e^{-\frac{t}{15}}=\boxed{e^{-\frac{3t}{20}}}$$
\subsection*{(c)}
Since these are possion processes, the sum of them would also be a possion process
with rate of $30+20+15+65$ seconds, so the probability density function 
for the time between events is an exponential with $\lambda=65$ so it would be:
$$\boxed{f(t)=65e^{-65t}}$$

\end{document}