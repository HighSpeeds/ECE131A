\include{"../preamble.tex"}
\title{ECE 131A HW 1}
\begin{document}
\maketitle
\section*{Problem 1}
\subsection*{(a)}
\begin{align*}
    P[B_0]&=P[A_0]P[B_0|A_0]+P[A_1]P[B_0|A_1]\\
    &=\boxed{\frac{1}{2}\cdot(1-\epsilon_1+\epsilon_2)}\\
\end{align*}
\subsection*{(b)}
We have 
$$P[B_1]=1-P[B_0]=\frac{1}{2}\cdot(\epsilon_1+1-\epsilon_2)$$
Therefore from Bayes law we have
\begin{align*}
    P[A_1|B_1]&=\frac{P[B_1|A_1]P[A_1]}{P[B_1]}\\
    &=\frac{1-\epsilon_2}{\epsilon_1+1-\epsilon_2}
\end{align*}
\begin{align*}
    P[A_0|B_1]&=\frac{P[B_1|A_0]P[A_0]}{P[B_1]}\\
    &=\frac{\epsilon_1}{\epsilon_1+1-\epsilon_2}
\end{align*}
Therefore we have for $\epsilon_1=0.25$ and $\epsilon_2=0.5$:
we will have
\begin{align*}
    P[A_1|B_1]&=\frac{1-0.5}{0.25+1-0.5}\\
    &=\frac{2}{3}\\
    P[A_0|B_1]&=\frac{0.25}{0.25+1-0.5}\\
    &=\frac{1}{3}
\end{align*}
Therefore $A_1$ will be more likely. 
\section*{Problem 2}
\subsection*{(a)}
\subsubsection*{(i)}
$$\boxed{20 \choose 15}$$
\subsubsection*{(ii)}
$$\boxed{15 \choose 4}$$
\subsection{(b)}
If we care about the order in which we place the balls in the buckets, the probability that we place all 5 balls in diffrent buckets is $\frac{1}{5^5}$.
There are $5!$ ways to order the balls to place into the buckets, so since we do not care about
the order in which we place the balls in the buckets, the probability that we place all 5 balls in diffrent buckets is $\frac{1}{5^5}\cdot5!=0.0384
$.
\subsection*{(c)}
This is a multnomial permutation so we have
$$\frac{9!}{4!2!3!}=\boxed{1260}$$

\section*{Problem 3}
\section*{Problem 4}
\section*{Problem 5}
\section*{Problem 6}
\subsection*{(a)}
Let the random event denoting whether we recived the correct answer 
be $C$ and the random event denoting whether we asked a Tourist be 
$T$. Then we have

$$P(C)=P(C|T)P(T)+P(C|T^c)P(T^c)$$
$$P(C)=\frac{2}{3}\frac{3}{4}=\boxed{\frac{1}{2}}$$
\subsection*{(b)}
Let the random event $A_1$ denote our first response, and $A_2$ denote our
second response we have that
\begin{align*}
    P(C|A_1=A_2)&=P(T|A_1=A_2)P(C|A_1=A_2,T)+P(T^c|A_1=A_2)P(C|A_1=A_2,T^c)
\end{align*}
Since 
\begin{align*}
    P(T|A_1=A_2)&=\frac{P(T)P(A_1=A_2|T)}{P(A_1=A_2)}\\
    &=\frac{P(T)P(A_1=A_2|T)}{P(A_1=A_2|T)P(T)+P(A_1=A_2|T^c)P(T^c)}\\
    &=\frac{\left(\frac{3^2}{4^2}+\frac{1}{4^2}\right)\cdot\frac{2}{3}}{\left(\frac{3^2}{4^2}+\frac{1}{4^2}\right)\cdot\frac{2}{3}+\frac{1}{3}}\\
    &=\frac{20}{36}
\end{align*}
And 
$$P(C|A_1=A_2,T)=\frac{\left(\frac{3}{4}\right)^2}{\left(\frac{3}{4}\right)^2+\left(\frac{1}{4}\right)^2}=\frac{9}{10}$$
and 
$$P(C|A_1=A_2,T^c)=0$$
Therefore we have
\begin{align*}
    P(C|A_1=A_2)&=\frac{20}{36}\cdot\frac{9}{10}+0\\
    &=\boxed{\frac{1}{2}}
\end{align*}
\subsection*{(c)}
Let the random event $A_3$ denote the third response we get
\begin{align*}
    P(C|A_1=A_2=A_3)&=P(T|A_1=A_2=A_3)P(C|A_1=A_2=A_3,T)+P(T^c|A_1=A_2=A_3)P(C|A_1=A_2=A_3,T^c)
\end{align*}
Since 
\begin{align*}
    P(T|A_1=A_2=A_3)&=\frac{P(T)P(A_1=A_2=A_3|T)}{P(A_1=A_2=A_3)}\\
    &=\frac{P(T)P(A_1=A_2=A_3|T)}{P(A_1=A_2=A_3|T)P(T)+P(A_1=A_2=A_3|T^c)P(T^c)}\\
    &=\frac{\left(\frac{3^3}{4^3}+\frac{1}{4^3}\right)\cdot\frac{2}{3}}{\left(\frac{3^3}{4^3}+\frac{1}{4^3}\right)\cdot\frac{2}{3}+\frac{1}{3}}\\
    &=\frac{56}{120}
\end{align*}
And 
$$P(C|A_1=A_2=A_3,T)=\frac{\left(\frac{3}{4}\right)^3}{\left(\frac{3}{4}\right)^3+\left(\frac{1}{4}\right)^3}=\frac{27}{28}$$
and 
$$P(C|A_1=A_2,T^c)=0$$
Therefore we have
\begin{align*}
    P(C|A_1=A_2)&=\frac{56}{120}\cdot\frac{27}{28}+0\\
    &=\boxed{\frac{9}{20}}
\end{align*}
\subsection*{(d)}
Let the random event $A_4$ denote the fourth response we get
\begin{align*}
    P(C|A_1=A_2=A_3=A_4)&=P(T|A_1=A_2=A_3=A_4)P(C|A_1=A_2=A_3=A_4,T)+P(T^c|A_1=A_2=A_3=A_4)P(C|A_1=A_2=A_3=A_4,T^c)
\end{align*}
Since
\begin{align*}
    P(T|A_1=A_2=A_3=A_4)&=\frac{P(T)P(A_1=A_2=A_3=A_4|T)}{P(A_1=A_2=A_3=A_4)}\\
    &=\frac{P(T)P(A_1=A_2=A_3=A_4|T)}{P(A_1=A_2=A_3=A_4|T)P(T)+P(A_1=A_2=A_3=A_4|T^c)P(T^c)}\\
    &=\frac{\left(\frac{3^4}{4^4}+\frac{1}{4^4}\right)\cdot\frac{2}{3}}{\left(\frac{3^4}{4^4}+\frac{1}{4^4}\right)\cdot\frac{2}{3}+\frac{1}{3}}\\
    &=\frac{82}{210}
\end{align*}
And
$$P(C|A_1=A_2=A_3=A+4,T)=\frac{\left(\frac{3}{4}\right)^4}{\left(\frac{3}{4}\right)^4+\left(\frac{1}{4}\right)^4}=\frac{81}{82}$$
and
$$P(C|A_1=A_2=A_3=A_4,T^c)=0$$
Therefore we have
\begin{align*}
    P(C|A_1=A_2=A_3=A_4)&=\frac{82}{210}\cdot\frac{81}{82}+0\\
    &=\boxed{\frac{27}{70}}
\end{align*}
\end{document}