\documentclass[12pt]{article}
\author{Lawrence Liu}
\usepackage{subcaption}
\usepackage{graphicx}
\usepackage{amsmath}
\usepackage{pdfpages}
\newcommand{\Laplace}{\mathscr{L}}
\setlength{\parskip}{\baselineskip}%
\setlength{\parindent}{0pt}%
\usepackage{xcolor}
\usepackage{listings}
\definecolor{backcolour}{rgb}{0.95,0.95,0.92}
\usepackage{amssymb}
\usepackage{empheq}

\newcommand*\widefbox[1]{\fbox{\hspace{2em}#1\hspace{2em}}}
\lstdefinestyle{mystyle}{
    backgroundcolor=\color{backcolour}}
\lstset{style=mystyle}

\title{ECE 131A HW 1}
\begin{document}
\maketitle
\section*{Problem 1}
\subsection*{(a)}
\begin{align*}
    P[B_0]&=P[A_0]P[B_0|A_0]+P[A_1]P[B_0|A_1]\\
    &=\boxed{\frac{1}{2}\cdot(1-\epsilon_1+\epsilon_2)}\\
\end{align*}
\subsection*{(b)}
We have 
$$P[B_1]=1-P[B_0]=\frac{1}{2}\cdot(\epsilon_1+1-\epsilon_2)$$
Therefore from Bayes law we have
\begin{align*}
    P[A_1|B_1]&=\frac{P[B_1|A_1]P[A_1]}{P[B_1]}\\
    &=\frac{1-\epsilon_2}{\epsilon_1+1-\epsilon_2}
\end{align*}
\begin{align*}
    P[A_0|B_1]&=\frac{P[B_1|A_0]P[A_0]}{P[B_1]}\\
    &=\frac{\epsilon_1}{\epsilon_1+1-\epsilon_2}
\end{align*}
Therefore we have for $\epsilon_1=0.25$ and $\epsilon_2=0.5$:
we will have
\begin{align*}
    P[A_1|B_1]&=\frac{1-0.5}{0.25+1-0.5}\\
    &=\frac{2}{3}\\
    P[A_0|B_1]&=\frac{0.25}{0.25+1-0.5}\\
    &=\frac{1}{3}
\end{align*}
Therefore $A_1$ will be more likely. 
\section*{Problem 2}
\subsection*{(a)}
\subsubsection*{(i)}
$$\boxed{19 \choose 15}$$
\subsubsection*{(ii)}
$$\boxed{15 \choose 4}$$
\subsection*{(b)}
If we care about the order in which we place the balls in the buckets, the probability that we place all 5 balls in diffrent buckets is $\frac{1}{5^5}$.
There are $5!$ ways to order the balls to place into the buckets, so since we do not care about
the order in which we place the balls in the buckets, the probability that we place all 5 balls in diffrent buckets is $\frac{1}{5^5}\cdot5!=0.0384
$.
\subsection*{(c)}
This is a multnomial permutation so we have
$$\frac{9!}{4!2!3!}=\boxed{1260}$$

\section*{Problem 3}
\subsection*{(a)}
There are $\frac{n!}{(n-r)!}$ ways to choose $r$ numbers from the first $n$
positive integers. Of these, there are $r!$ ways to order the $r$ numbers
such that their order is increasing. Therefore the probability the numbers are choosen in increasing order
is just $\boxed{\frac{1}{r!}}$.
\subsection*{(b)}
We have that the there are $n\choose r$ ways to choose
$r$ numbers from the first $n$ positive integers if we 
do not care about order.
Of these, there is only 1 way for L and our choice to have the same
numbers, so the probability is: $\boxed{\frac{1}{{n\choose r}}}$
\subsection*{(c)}
This is a hypergeometric distribution, with $k=k$, $n=r$, $K=r$, $N=n$, 
so we have the probabiblity of that is:
$$\boxed{\frac{{r\choose k}{n-r\choose r-k}}{{n\choose r}}}$$
\section*{Problem 4}
\subsection*{(a)}
Let $p$ be the probability that A is not hit, then we have
$$p=p_A(1-p_B)+(1-p_B)(1-p_A)p$$
$$p(1-(1-p_B)(1-p_A))=(1-p_A)p_B$$
$$p=\boxed{\frac{(1-p_B)p_A}{1-(1-p_B)(1-p_A)}}$$
\subsection*{(b)}
Let probability that the duel ends with both being hit be $p_{AB}$
We have:
\begin{align*}
    p_{AB}=p_Ap_B+(1-p_A)(1-p_B)p_{AB}\\
    p_{AB}\left(1-(1-p_A)(1-p_B)\right)=p_Ap_B\\
    p_{AB}=\boxed{\frac{p_Ap_B}{1-(1-p_A)(1-p_B)}}
\end{align*}
\subsection*{(c)}
The probabiblity is that the dual ends after n rounds, is the probability
that the two dualists do not hit each other for $n-1$ rounds, so $\left((1-p_A)(1-p_B)\right)^{n-1}$,
multiplied by the probabiblity that one or both get hit at the $n$ th round
so $1-(1-p_A)(1-p_B)$. Therefore the total probabiblity is 
$$\boxed{\left((1-p_A)(1-p_B)\right)^{n-1}\cdot(1-(1-p_A)(1-p_B))}$$
\subsection*{(d)}
Let $N$ be the random variable representing the number of rounds
of shots until the duel ends. Furthermore let $H_A$ be the event that A is not hit. 
From part (a) we have $P(H_A)=\frac{(1-p_B)p_A}{1-(1-p_B)(1-p_A)}$
we have
\begin{align*}
    P(N=n|H_A)&=\frac{P(N=n,H_A)}{P(H_A)}\\
    &=\frac{\left((1-p_A)(1-p_B)\right)^{n-1}p_A(1-p_B)}{\frac{(1-p_B)p_A}{1-(1-p_B)(1-p_A)}}\\
    &=\boxed{\left((1-p_A)(1-p_B)\right)^{n-1}\left(1-(1-p_B)(1-p_A)\right)}
\end{align*}
\subsection*{(e)}
Let $N$ be the random variable that representing the number of rounds
of shots until the duel ends. Furthermore let $H_{AB}$ be the event that A and B are hit, we have
\begin{align*}
    P(H_{AB}|N=n)&=\frac{\left((1-p_A)(1-p_B)\right)^{n-1}p_Ap_B}{\left((1-p_A)(1-p_B)\right)^{n-1}\cdot(1-(1-p_A)(1-p_B))}\\
    &=\boxed{\frac{p_Ap_B}{1-(1-p_B)(1-p_A)}}
\end{align*}

\section*{Problem 5}
\subsection*{(a)}
There are $10\choose 3$ to choose 3 socks from 10 if we do not
care about the color of the socks we pick. If we do care about
not selecting a pair of socks with the same colors we have
$2^3{5\choose 3}$ ways to choose 3 socks from 10.
therefore the probability is $\frac{2^3{5\choose 3}}{{10\choose 3}}=\boxed{0.66666}$.
\subsection*{(b)}
There are $2s \choose r$ ways to choose $r$ socks from $2s$ if we do not care about the color of the socks we pick. If we do care about
selecting the socks such that no two have the same color we have that we can do that
in $2^r{s\choose r}$ ways. 
Therefore the probability is $\boxed{\frac{2^r{s\choose r}}{{2s \choose r}}}$.
\section*{Problem 6}
\subsection*{(a)}
Let the random event denoting whether we recived the correct answer 
be $C$ and the random event denoting whether we asked a Tourist be 
$T$. Then we have

$$P(C)=P(C|T)P(T)+P(C|T^c)P(T^c)$$
$$P(C)=\frac{2}{3}\frac{3}{4}=\boxed{\frac{1}{2}}$$
\subsection*{(b)}
Let the random event $A_1$ denote our first response, and $A_2$ denote our
second response we have that
\begin{align*}
    P(C|A_1=A_2)&=P(T|A_1=A_2)P(C|A_1=A_2,T)+P(T^c|A_1=A_2)P(C|A_1=A_2,T^c)
\end{align*}
Since 
\begin{align*}
    P(T|A_1=A_2)&=\frac{P(T)P(A_1=A_2|T)}{P(A_1=A_2)}\\
    &=\frac{P(T)P(A_1=A_2|T)}{P(A_1=A_2|T)P(T)+P(A_1=A_2|T^c)P(T^c)}\\
    &=\frac{\left(\frac{3^2}{4^2}+\frac{1}{4^2}\right)\cdot\frac{2}{3}}{\left(\frac{3^2}{4^2}+\frac{1}{4^2}\right)\cdot\frac{2}{3}+\frac{1}{3}}\\
    &=\frac{20}{36}
\end{align*}
And 
$$P(C|A_1=A_2,T)=\frac{\left(\frac{3}{4}\right)^2}{\left(\frac{3}{4}\right)^2+\left(\frac{1}{4}\right)^2}=\frac{9}{10}$$
and 
$$P(C|A_1=A_2,T^c)=0$$
Therefore we have
\begin{align*}
    P(C|A_1=A_2)&=\frac{20}{36}\cdot\frac{9}{10}+0\\
    &=\boxed{\frac{1}{2}}
\end{align*}
\subsection*{(c)}
Let the random event $A_3$ denote the third response we get
\begin{align*}
    P(C|A_1=A_2=A_3)&=P(T|A_1=A_2=A_3)P(C|A_1=A_2=A_3,T)+P(T^c|A_1=A_2=A_3)P(C|A_1=A_2=A_3,T^c)
\end{align*}
Since 
\begin{align*}
    P(T|A_1=A_2=A_3)&=\frac{P(T)P(A_1=A_2=A_3|T)}{P(A_1=A_2=A_3)}\\
    &=\frac{P(T)P(A_1=A_2=A_3|T)}{P(A_1=A_2=A_3|T)P(T)+P(A_1=A_2=A_3|T^c)P(T^c)}\\
    &=\frac{\left(\frac{3^3}{4^3}+\frac{1}{4^3}\right)\cdot\frac{2}{3}}{\left(\frac{3^3}{4^3}+\frac{1}{4^3}\right)\cdot\frac{2}{3}+\frac{1}{3}}\\
    &=\frac{56}{120}
\end{align*}
And 
$$P(C|A_1=A_2=A_3,T)=\frac{\left(\frac{3}{4}\right)^3}{\left(\frac{3}{4}\right)^3+\left(\frac{1}{4}\right)^3}=\frac{27}{28}$$
and 
$$P(C|A_1=A_2,T^c)=0$$
Therefore we have
\begin{align*}
    P(C|A_1=A_2)&=\frac{56}{120}\cdot\frac{27}{28}+0\\
    &=\boxed{\frac{9}{20}}
\end{align*}
\subsection*{(d)}
Let the random event $A_4$ denote the fourth response we get
\begin{align*}
    P(C|A_1=A_2=A_3=A_4)&=P(T|A_1=A_2=A_3=A_4)P(C|A_1=A_2=A_3=A_4,T)+P(T^c|A_1=A_2=A_3=A_4)P(C|A_1=A_2=A_3=A_4,T^c)
\end{align*}
Since
\begin{align*}
    P(T|A_1=A_2=A_3=A_4)&=\frac{P(T)P(A_1=A_2=A_3=A_4|T)}{P(A_1=A_2=A_3=A_4)}\\
    &=\frac{P(T)P(A_1=A_2=A_3=A_4|T)}{P(A_1=A_2=A_3=A_4|T)P(T)+P(A_1=A_2=A_3=A_4|T^c)P(T^c)}\\
    &=\frac{\left(\frac{3^4}{4^4}+\frac{1}{4^4}\right)\cdot\frac{2}{3}}{\left(\frac{3^4}{4^4}+\frac{1}{4^4}\right)\cdot\frac{2}{3}+\frac{1}{3}}\\
    &=\frac{82}{210}
\end{align*}
And
$$P(C|A_1=A_2=A_3=A+4,T)=\frac{\left(\frac{3}{4}\right)^4}{\left(\frac{3}{4}\right)^4+\left(\frac{1}{4}\right)^4}=\frac{81}{82}$$
and
$$P(C|A_1=A_2=A_3=A_4,T^c)=0$$
Therefore we have
\begin{align*}
    P(C|A_1=A_2=A_3=A_4)&=\frac{82}{210}\cdot\frac{81}{82}+0\\
    &=\boxed{\frac{27}{70}}
\end{align*}
\end{document}