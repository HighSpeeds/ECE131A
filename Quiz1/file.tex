\include{"../preamble.tex"}
\title{ECE 131A Quiz 1}
\begin{document}
\maketitle
\section*{Problem 1}
\subsection*{(a)}
Since $P(A^c\cap B^c)=\frac{2}{3}$, we have 
$$P((A\cap B^c)\cup (A^c\cap B)\cup (A\cap B))=1-P(A^c\cap B^c)=\boxed{\frac{1}{3}}$$.
\subsection*{(b)}
We have
$$P(C\cap D )=P(C,D)$$
$$P(C|D)=\frac{P(C,D)}{P(D)}$$
$$P(C^c|D)=1-\frac{P(C,D)}{P(D)}$$
$$P(C^c\cap D )=P(C^c,D)=P(D)P(C^c|D)=P(D)(1-\frac{P(C,D)}{P(D)})$$
$$P(C^c\cap D )=P(D)-P(C,D)=0.45-0.1=\boxed{0.35}$$
\subsection*{(c)}
We have
$$P(A\cup B)=1-P\left((A\cup B)^c\right)=0.58$$
Furthermore we have 
$$P(A\cup B)=P(A)+P(B)-P(A\cap B)=0.12$$
This is equal to $P(A)P(B)=0.12$, therefore $A$ and $B$ are $\boxed{\text{independent}}$.

\section*{Problem 2}
Let the bernoulli random variable $T$ reprsent if a page from a 
textbook have a theorem on it. Likewise let $M$ be a  random variable that represents the type of 
textbook a student is reading given it is a math textbook, therefore we have
$$P(M=\text{probability})=\frac{10}{100-72}$$
$$P(M=\text{algebra})=\frac{13}{100-72}$$
$$P(M=\text{real analysis})=\frac{5}{100-72}$$
And thus we have
\begin{align*}
P(M=\text{probability}|T=1)&=\frac{P(M=\text{probability})P(T=1|M=\text{probability})}{P(T=1)}
\end{align*}
Since
\begin{align*}
    P(T=1)=&P(M=\text{probability})P(T=1|M=\text{probability})
\\&+P(M=\text{algebra})P(T=1|M=\text{algebra})\\&+
P(M=\text{real analysis})P(T=1|M=\text{real analysis})\\
=&\frac{10}{100-72}\frac{20}{100}+\frac{13}{100-72}\frac{29}{100}+\frac{5}{100-72}\frac{37}{100}\\
=&0.272142857143
\end{align*}
Therefore we have
\begin{align*}
    P(M=\text{probability}|T=1)&=\frac{P(M=\text{probability})P(T=1|M=\text{probability})}{P(T=1)}\\
    &=\frac{\frac{10}{100-72}\frac{20}{100}}{0.272142857143}\\
    &=\boxed{0.262467191601}
\end{align*}
\section*{Problem 3}
The probability of getting a speeding ticket (denoted by a bernoulli random variable $T_1$) from 
$L_1$ is:
$$P(T_1=1)=0.2\cdot0.4$$
Likewise the probability of getting a speeding ticket from $L_2$ is:
$$P(T_2=1)=0.1\cdot0.3$$
The probability of getting a speeding ticket from $L_3$ is:
$$P(T_3=1)=0.5\cdot0.2$$
The probability of getting a speeding ticket from $L_4$ is:
$$P(T_4=1)=0.3\cdot0.2$$
Therefore the probability of getting a speeding ticket is:
$$P(T_1=0)+P(T_2=0)+P(T_3=0)+P(T_4=0)=0.2\cdot0.4+0.1\cdot0.3+0.5\cdot0.2+0.3\cdot0.2=\boxed{0.27}$$


\end{document}