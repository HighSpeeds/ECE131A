\include{"../preamble.tex"}
\title{ECE 131A HW 1}
\begin{document}
\maketitle
\section*{Problem 1}
\subsection*{(a)}
The sample space is just all the combinations of the two dice, with $r$ denoting the 
bottom face of the red die and $g$ denoting the bottom face of the green die. Therefore the sample space
is just all the 16 orderd sets possible combinations of the possible results of $r$ and $g$. In other words, the sample space is

\begin{empheq}[box=\widefbox]{align*}
    \{
        (1,1), (1,2), (1,3), (1,4),& (2,1), (2,2), (2,3), (2,4), \\
        (3,1), (3,2), (3,3), (3,4),& (4,1), (4,2), (4,3), (4,4)
    \}  
\end{empheq}
\subsection*{(b)}
\subsubsection*{(i)}
$$E=\boxed{\{(1, 1), (1, 3), (3, 1), (3, 3)\}}$$
$$F=\boxed{\{(1, 4), (2, 3), (3, 2), (4, 1)\}}$$
\subsubsection*{(ii)}
Since $E$ requires all values to be odd, and thus their sum to be even
and $F$ requires the sum of the values to be odd (5), these sets are disjoint,
thus:
$$E\cap F=\boxed{\emptyset}$$
\subsubsection*{(iii)}
$$E\cup F=\boxed{\{(1, 1), (1, 3), (3, 1), (3, 3), (1, 4), (2, 3), (3, 2), (4, 1)\}}$$
\subsubsection*{(iv)}
$G^c$ is effectively just all value sof $g$ and $g$ such that
$g+r\geq 7$, thus
$$G^{c}=\boxed{\{(3, 4), (4, 3), (4, 4)\}}$$

\end{document}