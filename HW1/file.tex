\documentclass[12pt]{article}
\author{Lawrence Liu}
\usepackage{subcaption}
\usepackage{graphicx}
\usepackage{amsmath}
\usepackage{pdfpages}
\newcommand{\Laplace}{\mathscr{L}}
\setlength{\parskip}{\baselineskip}%
\setlength{\parindent}{0pt}%
\usepackage{xcolor}
\usepackage{listings}
\definecolor{backcolour}{rgb}{0.95,0.95,0.92}
\usepackage{amssymb}
\usepackage{empheq}

\newcommand*\widefbox[1]{\fbox{\hspace{2em}#1\hspace{2em}}}
\lstdefinestyle{mystyle}{
    backgroundcolor=\color{backcolour}}
\lstset{style=mystyle}

\title{ECE 131A HW 1}
\begin{document}
\maketitle
\section*{Problem 1}
\subsection*{(a)}
The sample space is just all the combinations of the two dice, with $r$ denoting the 
bottom face of the red die and $g$ denoting the bottom face of the green die. Therefore the sample space
is just all the 16 orderd sets possible combinations of the possible results of $r$ and $g$. In other words, the sample space is

\begin{empheq}[box=\widefbox]{align*}
    \{
        (1,1), (1,2), (1,3), (1,4),& (2,1), (2,2), (2,3), (2,4), \\
        (3,1), (3,2), (3,3), (3,4),& (4,1), (4,2), (4,3), (4,4)
    \}  
\end{empheq}
\subsection*{(b)}
\subsubsection*{(i)}
$$E=\boxed{\{(1, 1), (1, 3), (3, 1), (3, 3)\}}$$
$$F=\boxed{\{(1, 4), (2, 3), (3, 2), (4, 1)\}}$$
\subsubsection*{(ii)}
Since $E$ requires all values to be odd, and thus their sum to be even
and $F$ requires the sum of the values to be odd (5), these sets are disjoint,
thus:
$$E\cap F=\boxed{\emptyset}$$
\subsubsection*{(iii)}
$$E\cup F=\boxed{\{(1, 1), (1, 3), (3, 1), (3, 3), (1, 4), (2, 3), (3, 2), (4, 1)\}}$$
\subsubsection*{(iv)}
$G^c$ is effectively just all value sof $g$ and $g$ such that
$g+r\geq 7$, thus
$$G^{c}=\boxed{\{(3, 4), (4, 3), (4, 4)\}}$$
\section*{Problem 2}
Since $P(A)+P(B)=\frac{13}{12}>1$ we have that the minimum for $P(A\cap B)$ is $1-P(A)-P(B)$, which is $\frac{1}{12}$. Likewise, $P(A\cap B)$ is maximized when A and B totally overlap. In this case
would be equal to the minimum of $P(A)$ or $P(B)$, so $P(A\cap B)=P(B)=\frac{1}{3}$.
\section*{Problem 3}
The total possibilities for choosing $M$ items from $100$ items is $100\choose M$. Therefore we have that
the probability $p$ that $m$ items are defective of the M chosen is 

$$p=\begin{cases}
    0 & \text{if } m>k\\
    \frac{{k\choose m}\cdot {(100-k)\choose M-m}}{{100\choose M}} & \text{if } m\leq k
\end{cases}$$
\section*{Problem 4}
\subsection*{(a)}
STATISTICS is a 10 letter word, therefore if every letter was unique, we would have
$10!$ possible ways to arrange the words. However not every letter is unique, there are 2 occurences of I, and
3 occurences of S, and 3 occurences of T. Therefore the number of possible ways to arrange the letters is 
$$\frac{10!}{2!3!3!}=\boxed{50400}$$
\subsection*{(b)}
We can effectively treat the two "I"s together as one letter, so then we would
have the total arrangements of the letters in STATISTICS with the two "I"s together 
is $\frac{9!}{3!3!}$. Therefore the probability of this occurence is 
$$\frac{\frac{9!}{3!3!}}{50400}=\boxed{\frac{1}{5}}$$
\section*{Problem 5}
\end{document}